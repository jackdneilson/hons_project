\documentclass{article}
\usepackage[margin=30mm]{geometry}
\usepackage[comma]{natbib}
\usepackage{todonotes}
\bibliographystyle{agsm}

\begin{document}
\title{Using Facial Recognition to gather Social Media Intelligence}
\author{Jack Neilson}
\maketitle
\newpage
\section{Literature Review}
\subsection{Background}
\subsubsection{SOCMINT}
Social media intelligence (SOCMINT) is an emergent field in intelligence gathering where data is gathered from social media profiles. Massive amounts of data are added to social media services every day \citep(socmintoverview), much of it personal, making social media sites a potentially valuable resource when gathering information about groups or individuals \citep{gchqmasssurveillance}. Social networks have also been used as a means of communication between persons of interest to the security services, making mining intelligence from their profiles a high priority \citep{socmintoverview}\citep{policesocmint}\citep{prismslides}.

As it stands, social media intelligence sources are woefully underutilised. After the 2011 riots in London that were organised in large part on social media, Her Majesty's Inspectorate of Constabulary stated that the police services were "insufficiently equipped" to effectively use SOCMINT in their response\citep{socmintpublicsafety}. This is not to say that the value of SOCMINT is not realised however, as many intelligence agencies are investing in tools to effectively gather and analyse SOCMINT \citep{socmintpublicsafety} or are performing case studies in to potential uses \citep{bostonbombingcasestudy}.

\subsubsection{Uses of SOCMINT}

\subsubsection{Current Applications}

\subsubsection{Facial Recognition}

\subsubsection{Uses of Facial Recognition}

\subsubsection{Constrained vs Unconstrained}

\subsection{Theory}
\subsubsection{Prior Work}

\subsubsection{Individual vs Group Data}

\subsubsection{Quantity of Information}

\subsubsection{Accessibility of Data}

\subsubsection{Uses}

\subsubsection{Challenges and Constraints}

\subsection{SOCMINT}
\subsubsection{Prior Knowledge Attacks}

\subsubsection{HUMINT}

\subsubsection{Social Engineering}

\subsubsection{Spearphish}

\subsection{Facial Recognition}
\todo{Overview and current applications}

\subsubsection{Challenges}

\subsubsection{Recent Advances}

\subsubsection{Unconstrained Facial Recognition}

\bibliography{citations}
\end{document}